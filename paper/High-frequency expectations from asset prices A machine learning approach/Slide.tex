\documentclass[UTF8, 16pt]{beamer}
 
% Chinese
\usepackage{CJKutf8}

% Font
\usepackage{bookman}
\usefonttheme{serif}
%\usepackage[T1]{fontenc}
%\usepackage{tgbonum}

% Other packages
\usepackage{appendixnumberbeamer}
\usepackage{latexsym, amsmath, xcolor, multicol, booktabs}
\usepackage{graphicx, listings, stackengine}
\usepackage{multirow}
\usepackage{hyperref}
\hypersetup{hidelinks,
	colorlinks=true,
	allcolors=black,
	pdfstartview=Fit,
	breaklinks=true}
% SUFE.sty
\usepackage{SUFE} 
% Bibtex
\usepackage[citestyle=authoryear-comp, 
			backend=bibtex, 
			bibstyle=numeric, 
%			sorting=ynt
			]{biblatex}
\setbeamertemplate{bibliography item}[text]
\addbibresource{ref.bib}

% Other setting


%%%%%%%%%%%%%%%%%%%%%%%%%%

% Title page
%% Author
\author[Haotian Deng] % The short name
{
Aditya Chaudhry and Sangmin S. Oh \\ \ \\
Presented by Haotian Deng
%邓皓天
%\inst{1}
%\and
%Yuting Liu 
%\inst{2}
} 
%% Title & Subtitle
\title[High-Frequency Expectations from Asset Prices: A Machine Learning Approach]{High-Frequency Expectations from Asset Prices: A Machine Learning Approach}
\subtitle{}
%% Institution
\institute[SUFE]
{
%\inst{1}
Shanghai University of Finance and Economics
%上海财经大学金融学院
%\and
%\inst{2}
%Shanghai University of Finance and Economics
}
%% Date
\date[VLC 2021]
{\today}
%%Logo
%\logo{\includegraphics[height=1cm]{sufe_logo}}

%%%%%%%%%%%%%%%%%%%%%%%%%%

% Document begins
\begin{document}
\begin{CJK*}{UTF8}{gbsn}

%Title page
\begin{frame}[noframenumbering]
%	\thispagestyle{empty}
	\titlepage
	% Logo
	\vspace{-0.5cm}
    \begin{figure}[htpb] 
        \begin{center}
            \includegraphics[width=0.19 \linewidth]{sufe_logo.png}
        \end{center}  
    \end{figure}
\end{frame}

% Contents page
\begin{frame}{Contents}
	\tableofcontents[sectionstyle=show,
 	subsectionstyle=show/shaded/hide,
 	subsubsectionstyle=show/shaded/hide]
\end{frame}

% Body
\section{Author}
\begin{frame}{Aditya Chaudhry}
	\begin{itemize}
		\item Publications
			\begin{enumerate}
				\item Uncertainty Assessment and False Discovery Rate Control in High-Dimensional Granger Causal Inference
			\end{enumerate}
		\item Working Papers
			\begin{enumerate}
				\item The Causal Impact of \alert{Macroeconomic Uncertainty} on Expected Returns
				\item How Much Do \alert{Subjective Growth Expectations} Matter for Asset Prices?
				\item The Impact of Prices on \alert{Analyst Cash Flow Expectations}
			\end{enumerate}
	\end{itemize}
	Primary research interests:
	\begin{itemize}
		\item Joint dynamics of \alert{subjective beliefs}, asset demand, and asset prices.
		\item Using \alert{new empirical strategies} to identify important structural parameters in asset pricing.
	\end{itemize}
\end{frame}
\begin{frame}{Sangmin S. Oh}
	\begin{itemize}
		\item Job Market Paper
			\begin{enumerate}
				\item Social Inflation
			\end{enumerate}
		\item Publications
			\begin{enumerate}
				\item Cross-sectional Skewness
			\end{enumerate}
		\item Working Papers
			\begin{enumerate}
				\item Pricing of Climate Risk Insurance: Regulation and Cross-Subsidies
				\item Asset Demand of U.S. Households
				\item Unpacking the Demand for Sustainable Equity Investing
				\item Climate Capitalists
			\end{enumerate}
	\end{itemize}
	
	\center Notebook: \href{https://sangmino.github.io/notebook}{https://sangmino.github.io/notebook}
\end{frame}

\section{Background}

\begin{frame}{Motivation: shortcomings of past approaches}
	\begin{itemize}
		\item Investors price assets based on their \alert{beliefs} about the joint distribution of \alert{stochastic discount factor $M_{t+1}$} and the \alert{asset’s cash flows $X_{t+1}$}. 
		$$P_{t}=\mathbb{E}_{t}\left[M_{t+1} X_{t+1}\right]$$
		\item \alert{Expectations} play a central role in asset pricing.
		\item One of the key drivers of investor \alert{expectations} is \alert{news of macroeconomic events}.
		\item How to test the impact of such \alert{news}?
			\begin{enumerate}
				\item Examined the behavior of asset prices around announcement dates.
					\\ \alert{Shortcomings: the diversity of information sources.}
				\item Utilized surveys that directly measure expectations.
					\\ \alert{Shortcomings: the low frequency of survey data.}
			\end{enumerate}
	\end{itemize}
\end{frame}

\begin{frame}{Aim: construct a daily time series of investor expectations}
	\begin{itemize}
		\item Task:  recover the unobserved \alert{daily series of expectations} between two quarterly survey releases dates.
		\item Previous papers: Kalman filtering (\alert{KF}) and a mixed frequency data sampling approach (\alert{MIDAS})
		\item In this paper: Reinforcement learning (\alert{RL}) utilized \alert{daily asset prices} that reflect investors’ \alert{updated beliefs} about \alert{macroeconomic growth}.
	\end{itemize}
	\center{Why RL? Why asset prices?}
\end{frame}

\begin{frame}{Why RL? RL achieves a significant gain in efficiency.}
	\begin{itemize}
		\item Observed series (asset prices)
			$y_{t+1}=H y_{t}+e_{t+1}, \quad e_{t+1} \sim \mathcal{N}(0, \Sigma)$
		\item Latent series (macroeconomic growth expectations)
			$x_{t+1}=F x_{t}+u_{t+1}, \quad u_{t+1} \sim \mathcal{N}(0, \Phi)$
	\end{itemize}
	Update rule for the estimate of x in the \alert{Kalman Filter} is
	$$\hat{x}_{t+1 \mid t}=F\left(\hat{x}_{t \mid t-1}+\left(\frac{H \Omega_{t \mid t-1}}{\Sigma+H^{2} \Omega_{t \mid t-1}}\right)\left(y_{t}-\hat{y}_{t}\right)\right)$$
	One must estimate the parameters $(H, F, \Sigma, \Phi)$ using ML, while \alert{RL} avoids this problem by estimating the update function directly: $\hat{x}_{t+1 \mid t}=\hat{x}_{t \mid t-1}+f\left(y_{t}\right)$.
	\begin{itemize}
		\item RL avoids an explicit model of the state dynamics and thus requires estimation of \alert{far fewer parameters}.
	\end{itemize}
\end{frame}

\begin{frame}{Why asset prices?}
	\begin{enumerate}
		\item Data must be available at a \alert{daily frequency}.
		\item Asset prices reflect many variables besides growth expectations.
			\begin{itemize}
				\item A single asset: cannot extract the component of asset returns driven \alert{solely} by changes in expectations of macroeconomic growth.
				\item With multiple assets: \alert{a suitable linear combination} can cancel the extraneous sources of return variation.
			\end{itemize}
	\end{enumerate}
	Task: finding an \alert{optimal combination of asset returns} that correlates maximally with the change \alert{investors' expectations of future macroeconomic growth}.
\end{frame}

\begin{frame}{Structure}
	\begin{enumerate}
		\item Providing empirical evidence regarding the \alert{relationship} between \alert{asset returns} and \alert{expectations of macroeconomic growth}.
		\item Elucidate the differences among \alert{RL} algorithm, \alert{KF} and \alert{MIDAS} regression by presenting \alert{a stylized economy} with \alert{Bayesian agents}.
		\item Take \alert{RL} algorithm into \alert{real data}.
		\item Use \alert{RL} estimated daily series of growth expectations to test the existence of \alert{the Fed information effect}.
	\end{enumerate}
\end{frame}

\section{Model of the economy}
\subsection{Campbell and Shiller Approximation}
\begin{frame}{Campbell and Shiller Approximation}
	$$
	\begin{aligned}
		R_{t+1}
		&=\small{\frac{P_{t+1}+D_{t+1}}{P_t}}
		=\frac{\frac{P_{t+1}}{D_{t+1}}+\frac{D_{t+1}}{D_{t+1}}}{\frac{P_t}{D_t}}\cdot \frac{D_{t+1}}{D_t}
		\\
		\frac{P_t}{D_t}&=\frac{1}{R_{t+1}}\cdot \left( 1+\frac{P_{t+1}}{D_{t+1}} \right) \cdot \frac{D_{t+1}}{D_t}
		\\
		\underbrace{\ln P_t}_{p_t}-\underbrace{\ln D_t}_{d_t}&=-\ln R_{t+1}+\ln \left( 1+\frac{P_{t+1}}{D_{t+1}} \right) +\ln \frac{D_{t+1}}{D_t}
		\\
		p_t-d_t&=-r_{t+1}+\ln{\left( 1+e^{p_{t+1}-d_{t+1}} \right)} +\varDelta d_{t+1}
	\end{aligned}
	$$
	Assume that $pd_{t}=\frac{1}{t}\sum_{n=1}^t{(p_i-d_i)}$, use Taylor expansion at $p_{t+1}-d_{t+1}=pd_{t}$, we have 
	$$
	\ln{\left( 1+e^{p_{t+1}-d_{t+1}} \right)}=\ln(1+e^{pd_t})+\frac{e^{pd_t}}{1+e^{pd_t}}(p_{t+1}-d_{t+1}-pd_t)
	$$
\end{frame}

\begin{frame}{Campbell and Shiller Approximation}
	$$
	\begin{aligned}
		p_t-d_t&=\left[ \ln \left( 1+e^{pd_t} \right) +\frac{e^{pd_t}}{1+e^{pd_t}}\left( p_{t+1}-d_{t+1}-pd_t \right) \right] 
		\\&-r_{t+1}+\varDelta d_{t+1}
		\\p_t-d_t&=-r_{t+1}+\varDelta d_{t+1}
		\\&+\underbrace{\frac{e^{pd_t}}{1+e^{pd_t}}}_{\rho}\left( p_{t+1}-d_{t+1} \right) +
		\underbrace{\left[ \ln \left( 1+e^{pd_t} \right) -\frac{e^{pd_t}}{1+e^{pd_t}}\cdot pd_t \right]}_{\kappa}
		\\p_t-d_t&=-r_{t+1}+\varDelta d_{t+1}+\rho \left( p_{t+1}-d_{t+1} \right) +\kappa
		\\p_t-d_t&=-\sum_{n=0}^{\infty}{\rho ^nr_{t+i+1}}+\sum_{n=0}^{\infty}{\rho ^n\varDelta d_{t+n+1}}+\rho ^{\infty}pd_{t+\infty}+\kappa \sum_{n=0}^{\infty}{\rho ^n}
	\end{aligned}
	$$
\end{frame}

\begin{frame}{Campbell and Shiller Approximation}
	$$
	\begin{aligned}
		p_t-d_t&=-\sum_{n=0}^{\infty}{\rho ^ir_{t+i+1}}+\sum_{n=0}^{\infty}{\rho ^i\varDelta d_{t+n+1}}+\rho ^{\infty}pd_{t+\infty}+\kappa \sum_{n=0}^{\infty}{\rho ^i}
		\\
		p_t&=\kappa \sum_{n=0}^{\infty}{\rho ^i}-\sum_{n=0}^{\infty}{\rho ^ir_{t+i+1}}
		\\&+\left[ d_t+\left( d_{t+1}-d_t \right) +\rho \left( d_{t+2}-d_{t+1} \right) +\cdots \right]
	\end{aligned}
	$$
	\begin{alertblock}{Campbell and Shiller Approximation}
		$$
		p_t=\frac{\kappa}{1-\rho}-\sum_{n=0}^{\infty}{\rho ^n r_{t+n+1}}+\left( 1-\rho \right) \sum_{n=0}^{\infty}{\rho^n d_{t+n+1}}
		$$
		\center{where $\rho=\frac{e^{pd_t}}{1+e^{pd_t}}$, $\kappa=\ln \left( 1+e^{pd_t} \right) -\frac{e^{pd_t}}{1+e^{pd_t}}\cdot pd_t$}
	\end{alertblock}
\end{frame}

\subsection{Model of the Economy}
\begin{frame}{State-space model}
	\begin{itemize}
		\item GDP growth is persistent 
			$$\theta_{t+1}  =\mu+\delta \theta_{t}+\epsilon_{t+1}, \quad \epsilon_{t+1} \sim N\left(0, \sigma_{\epsilon}^{2}\right)$$
		\item GDP growth affects each asset’s dividend growth
			$$d_{t+1}^{i}-d_{t}^{i}  =\gamma+\beta^{i} \theta_{t+1}+\nu_{t+1}^{i}, \quad \nu_{t+1}^{i} \sim N\left(0, \sigma_{\nu}^{2}\right)$$
		\item The conditional expected return of asset $i$ depends linearly on another latent factor $\zeta_{t}$
			$$\mathbb{E}_{t}\left[r_{t+1}^{i}\right]  =\alpha+\phi^{i} \zeta_{t}$$
		\item Latent factor $\zeta_{t}$ is persistent
			$$\zeta_{t+1}  =\tau+\psi \zeta_{t}+\xi_{t+1}, \quad \xi_{t+1} \sim N\left(0, \sigma_{\xi}^{2}\right)$$
		\item Innovations to $\theta_t$ and $\zeta_t$ are correlated:  $\operatorname{Corr}\left(\epsilon_{t}, \zeta_t\right)=\pi$
	\end{itemize}
\end{frame}
\begin{frame}{Solve the state-space model}
	\begin{alertblock}{Campbell and Shiller Approximation}
	$$
	p_{t}^{i} =\frac{\kappa}{1-\rho}+(1-\rho) \sum_{j=0}^{\infty} \rho^{j} \mathbb{E}_{t}\left[d_{t+j+1}^{i}\right]-\sum_{j=0}^{\infty} \rho^{j} \mathbb{E}_{t}\left[r_{t+j+1}^{i}\right]
	$$
	\end{alertblock}
	$$
	\begin{array}{c}
		\begin{cases}
			\theta_{t+1}  =\mu+\delta \theta_{t}+\epsilon_{t+1}
			\\
			d_{t+1}^{i}-d_{t}^{i}  =\gamma+\beta^{i} \theta_{t+1}+\nu_{t+1}^{i}
		\end{cases}
		\\
		\Downarrow
		\\
		\mathbb{E}_{t}\left[d_{t+j+1}^{i}\right]=d_{t}^{i}+\sum_{n=0}^{j}{\left[\gamma+\beta^{i} \mu\left(\frac{1-\delta^{n+1}}{1-\delta}\right)+\beta^{i} \delta^{n+1} \theta_{t}\right]}
	\end{array}
	$$
	$$\mathbb{E}_{t}\left[r_{t+1}^{i}\right]  =\alpha+\phi^{i} \zeta_{t} \Rightarrow \mathbb{E}_{t}\left[r_{t+j+1}^{i}\right]=\left\{\begin{array}{ll}\alpha+\phi^{i} \tau \frac{1-\psi^{j}}{1-\psi}+\phi^{i} \psi^{j} \zeta_{t} & j>1 \\ \alpha+\phi^{i} \zeta_{t} & j=1\end{array}\right.$$
\end{frame}
\begin{frame}{Model result}
	\begin{alertblock}{Simple function of return $r^{i}_{t+1}$}
		$$
		\begin{aligned}
			r_{t+1}^{i}&=\left[\left(\beta^{i}+\frac{\delta \beta^{i}}{1-\rho \delta}\right) \theta_{t+1}-\left(\frac{\delta \beta^{i}}{1-\rho \delta}\right) \theta_{t}\right]
		\\&-\frac{\phi^{i}}{1-\rho \psi}\left(\zeta_{t+1}-\zeta_{t}\right)+\nu_{t+1}+\gamma
		\end{aligned}
		$$
	\end{alertblock}
%	\\ \ \\
	\begin{columns}
		\column{0.59\textwidth}
			Returns increase with 
			\begin{itemize}
				\item contemporaneous growth $\theta_{t+1}$
				\item shock to the dividend process
			\end{itemize}
			and decrease with
			\begin{itemize}
				\item previous period's growth $\theta_t$
				\item change in $\zeta_{t+1}$
			\end{itemize}
		\column{0.45\textwidth}
		\alert{Asset prices}
		\\ should be useful to 
		\\ understand changes in 
		\\ \alert{investor expectations}
			\\ (GDP growth expectation)
	\end{columns}
\end{frame}

\section{Empirical framework}
\subsection{Estimate the cross-sectional moments of growth forecasts}
\begin{frame}{Asset Prices $\rightarrow$ Growth Expectations}
	Examine whether \alert{asset returns} \\ can explain \alert{innovations in the average growth forecast}.
	\begin{enumerate}
		\item Forecast innovation: the difference between the \alert{nowcast} and the \alert{lag-one-period forecast} for period $t$.
		\item Run time-series regressions of \alert{innovations in mean growth expectations} on \alert{asset returns} (bivariate pairs of assets).
		\item The CRSP U.S. Treasury five-year fixed-term index and the CRSP value-weighted portfolio ($R^2=38.3\%$).
	\end{enumerate}
	Thus, \alert{asset returns} contain useful information about \alert{forecast innovations} empirically.
\end{frame}
\begin{frame}{Incorporating Bayesian Agents}
	Instantiate 20 \alert{Bayesian agents} who observe realized returns and form expectations of the latent growth process (cannot observe growth).
	\begin{itemize}
		\item \alert{prior-mean heterogeneity}: the mean of each agent’s \alert{prior belief} regarding $\theta_t\sim N(\theta_0, 0.5\theta_0)$ at the start of the quarter.
		\item \alert{learning heterogeneity}: each agents draws his value of the parameters from a normal distribution centered at the baseline parameter value with variance parameterized by a fixed signal-to-noise ratio (\alert{different parameters in state and observation equation}).
	\end{itemize}
\end{frame}
\begin{frame}{Learning the Cross-sectional Moments}
	The expression for the optimal Kalman gain implies the following relationship:
	$$\underbrace{\mu_{i, t}}_{\mathbb{E}_{t}^{i}\left[\theta_{t+1}\right]}=c_{0, t}^{i}+c_{1, t}^{i} \mu_{i, t-1}+\left(\mathbf{c}_{2, t}^{i}\right)^{\prime} \mathbf{r}_{t}$$
	Averaging across all agents, we get the cross-sectional mean of growth expectations at period $t$:
	$$
	\mu_{t} \equiv \frac{1}{N} \sum_{i=1}^{N} \mu_{i, t}=\frac{1}{N} \sum_{i=1}^{N} c_{0, t}^{i}+\frac{1}{N} \sum_{i=1}^{N} c_{1, t}^{i} \mu_{i, t-1}+\left(\frac{1}{N} \sum_{i=1}^{N} \mathbf{c}_{2, t}^{i}\right)^{\prime} \mathbf{r}_{t}
	$$
	Use the following approximating moment:
	$$
	\begin{aligned}
		\mu_{t}&=c_{0}+c_{1} \mu_{t-1}+\mathbf{c}_{2}^{\prime} \mathbf{r}_{t} \approx c_{1} \mu_{t-1}+\mathbf{c}_{2}^{\prime} \mathbf{r}_{t}
		\\
		& =c_{1} \mu_{t-1}+\mathbf{c}_{2}^{\prime}\left[\mathbf{1} \gamma+\mathbf{a} \theta_{t}+\mathbf{b} \theta_{t-1}+\mathbf{c}\left(\zeta_{t}-\zeta_{t-1}\right)+\boldsymbol{\nu}_{t}\right]
	\end{aligned}
	$$
\end{frame}

\subsection{Three Approaches to Estimation: KF, RL, and MIDAS}
\begin{frame}{The Kalman Filtering (KF) Approach}
	State equation:
	$$
	\begin{aligned} \theta_{t+1} & =\mu+\delta \theta_{t}+\epsilon_{t+1} \\ \zeta_{t+1} & =\tau+\psi \zeta_{t}+\xi_{t+1} \\ \mu_{t+1} & =\mathbf{c}_{2}^{\prime}(\mathbf{1} \gamma+\mathbf{a} \mu+\mathbf{c} \tau)+\mathbf{c}_{2}^{\prime}(\mathbf{a} \delta+\mathbf{b}) \theta_{t}+\mathbf{c}_{2}^{\prime} \mathbf{c}(\psi-1) \zeta_{t}+c_{1} \mu_{t}\end{aligned}
	$$
	\\ \ \\
	Observation equation:
	$$
	\mathbf{c}_{2}^{\prime} \mathbf{r}_{t}=\mu_{t}-c_{1} \mu_{t-1}
	$$
	\begin{itemize}
		\item $3m + 11$ parameters to be estimated. \\ ($m$ is the number of assets used)
	\end{itemize}
\end{frame}
\begin{frame}{The Mixed Data Sampling (MIDAS) Approach}
	$$y_{t}=\alpha^{\tau}+\rho^{\tau} y_{t-1}+\sum_{i=1}^{m} {\beta_{i}^{\tau} \underbrace{\gamma^{\tau}(L) r_{\tau}^{i}}_{\sum_{d=\tau-l+1}^{\tau}{\gamma_{d}^{\tau} r_{d}^{i}}}}+\epsilon_{t}$$
	\begin{itemize}
		\item Use a maximal lag of $l = 90$ days.
		\item $y_t$ is $\mu_t$, the quarterly observed cross-sectional mean survey expectation.
		\item Each MIDAS regression involves estimating $m+4$ parameters.
	\end{itemize}
	
\end{frame}
\begin{frame}{The Reinforcement Learning (RL) Approach}
	\begin{itemize}
		\item agent's state = current expectation + asset returns
			$$
			\varphi\left(s_{t}\right)=\left(\begin{array}{c}\hat{\mu}_{t-1} \\ \hat{\sigma}_{t-1}^{2} \\ \mathbf{r}_{t}^{\prime}\end{array}\right) \in \mathbb{R}^{m+2}, \quad 
			\varphi\left(s_{1}\right)=\left(\begin{array}{c}\mu_{0} \\ \sigma_{0}^{2} \\ \mathbf{r}_{1}^{\prime}\end{array}\right)
			$$
		\item policy: function of the current state that yields the agent's new growth expectation.
			$$
			g_{\boldsymbol{\lambda}}\left(s_{t}\right) \equiv\left(\begin{array}{c}\mu_{t} \\ \sigma_{t}\end{array}\right)=\left(\begin{array}{c}c_{1} \mu_{t-1}+\mathbf{c}_{2}^{\prime} \mathbf{r}_{t} \\ \sqrt{c_{3} \sigma_{t-1}^{2}+\mathbf{c}_{4}^{\prime} \mathbf{r}_{t} \mathbf{r}_{t}^{\prime} \mathbf{c}_{4}+\mathbf{c}_{5}^{\prime} \mathbf{r}_{t} \mu_{t-1}}\end{array}\right) \in \mathbb{R}^{2}
			$$
		\item action: agent's updated growth expectation.
		\item rewards:
			$$r_{t}\left(s^{t}\right)=\left\{\begin{array}{ll}0 & \text { if } t<T \\ -\left\|\left(\begin{array}{c}\hat{\mu}_{T \mid T-1} \\ \hat{\sigma}_{T \mid T-1}\end{array}\right)-\left(\begin{array}{c}\mu_{T} \\ \sigma_{T}\end{array}\right)\right\| & \text { if } t=T\end{array}\right.$$
	\end{itemize}
\end{frame}
\begin{frame}{A Three-Way Comparison}
	\begin{itemize}
		\item The interpretation of the output of each method.
			\begin{itemize}
				\item RL and KF approaches yield daily estimates of the current latent cross-sectional mean expectation. $(\mathbb{E}\left[\mu_{t} \mid \mathcal{F}_{t}^{E}\right])$
				\item MIDAS produces a prediction of the end-of-quarter cross-sectional mean expectation.$(\mathbb{E}\left[\mu_{T} \mid \mathcal{F}_{t}^{E}\right])$
				\item RL and KF approaches prove better suited to our setting than the MIDAS approach.
			\end{itemize}
		\item The bias-variance tradeoff each method incurs.
			\begin{itemize}
				\item parameters\\
					KF: $3m+11$, RL: $m+1$, MIDAS: $60(m+4)$
				\item RL approach proves far more efficient than the other two methods.
			\end{itemize}
	\end{itemize}
\end{frame}
\subsection{Empirical Performance of RL}
\begin{frame}{Performance of RL}
	Policy function:
	$$
	g_{\boldsymbol{\lambda}}\left(s_{t}\right) \equiv\left(\begin{array}{c}\mu_{t} \\ \sigma_{t}\end{array}\right)=\left(\begin{array}{c}c_{1} \mu_{t-1}+\mathbf{c}_{2}^{\prime} \mathbf{r}_{t} \\ \sqrt{c_{3} \sigma_{t-1}^{2}+\mathbf{c}_{4}^{\prime} \mathbf{r}_{t} \mathbf{r}_{t}^{\prime} \mathbf{c}_{4}+\mathbf{c}_{5}^{\prime} \mathbf{r}_{t} \mu_{t-1}}\end{array}\right) \in \mathbb{R}^{2}
	$$
	\begin{table}
	    \centering
	    \caption{Recursive Out-of-Sample Estimation Results}
	    \vspace{-0.5cm}
	    \setlength{\tabcolsep}{4mm}
		    {
		    \begin{tabular}{lcccc}
		    \toprule
	         & RL Approach & Naive & MIDAS & KF \\ \midrule
	        RMSE & 0.449 & 0.588 & 0.916 & 39.103 \\ 
	        $R^2$ & 0.823 & 0.647 & 0.392 & 0.0237 \\ \bottomrule
		    \end{tabular}
		    }
%	    \label{fig1}
	\end{table}
\end{frame}
\begin{frame}{Origins of RL’s Outperformance}
	Core difficulty: obtaining a daily law of motion for expectations given quarterly training data.
	\begin{itemize}
		\item RL vs. KF
			\begin{itemize}
				\item KF: imposing parametric assumptions and using ML.
				\item RL: directly estimating the Kalman gain using a linear learning rule (\alert{bias-efficiency trade-off}).
			\end{itemize}
		\item RL vs. MIDAS
			\begin{itemize}
				\item MIDAS: applies a non-monotonic weighting scheme to 90 days of lagged asset returns.
				\item RL: uses only asset returns since the start of the last survey release, weighting them uniformly (\alert{treatment of lagged asset returns proves more useful}).
			\end{itemize}
	\end{itemize}
\end{frame}
\begin{frame}{Hyper-parameters: step size and noise in behavioral policy}
	\begin{itemize}
		\item step size: 
			\\ too small $\rightarrow$ one may get stuck in a local maximum
			\\ too large $\rightarrow$ algorithm may have trouble converging
		\item noise in the behavioral policy: 
			\\ too little exploration $\rightarrow$ a suboptimal policy
			\\ too much exploration $\rightarrow$ prevent the algorithm from making proper gradient updates to the weights
	\end{itemize}
	A proper hyper-parameter optimization procedure
	\begin{enumerate}
		\item Divide the sample into a \alert{training subsample} and a \alert{pseudo-testing subsample}.
		\item Train a model at each grid point on the training subsample and test on the pseudo-testing subsample.
		\item Choose the set of hyper-parameters that performs best in the pseudo-testing subsample.
	\end{enumerate}
\end{frame}
\subsection{Testing the "Fed Information Effect"}
\begin{frame}{Testing the "Fed Information Effect"}
	Fed Information Effect:\\ \alert{Hawkish surprises} for interest rates correspond to \alert{increases} in real GDP growth expectations.
	$$
	\mathbb{E}_{t+15}\left[g_{Q}\right]-\mathbb{E}_{t-15}\left[g_{Q}\right]=\beta_{0}+\underbrace{\beta_{1}}_{positive} \text{Shock}_{t}+\epsilon_{t}
	$$
	An omitted variable: \\
	\alert{economic news} released between day $t-15$ and day $t-1$
	$$
	\Delta CX\ Mean _{t}=\beta_{0}+\underbrace{\beta_{1}}_{negative} \text{Shock}_{t}+\epsilon_{t}
	$$
\end{frame}

\section{Conclusion}
\begin{frame}{Attribution}
	\begin{itemize}
		\item The first serious application of \alert{reinforcement learning} in the growing literature that uses machine learning methods in finance.
		\item Present \alert{reinforcement learning} as a more efficient improvement over \alert{traditional filtering} methods.
		\item Obtain a \alert{daily series of expectations} for any macroeconomic variable with a \alert{low-frequency} panel of forecasts.
	\end{itemize}
\end{frame}
\begin{frame}{How to apply "Machine Learning Approach" in finance?}
	\begin{itemize}
		\item Compare to traditional methods.
		\item A model of the economy.
		\item Economic intuition.
		\item Use the result to test something.
	\end{itemize}
\end{frame}
%%%%%%%%%%%%%%%%%%%%%%%%%%
% End
\begin{frame}[allowframebreaks]%{End}
	\begin{center}
		\Huge\textbf{\textit{\texttt{Thanks!}}}
	\end{center}
\end{frame}

% Reference
%\appendix
%\begin{frame}{Reference}
%	\addtocounter{framenumber}{-1}
%	\printbibliography % [heading=bibintoc, title=Reference]
%\end{frame}

%%%%%%%%%%%%%%%%%%%%%%%%%%

% Snippets
%\begin{frame}[noframenumbering, plain]{Snippets}
%	\begin{multicols}{2}
%		\begin{enumerate}
%			\item \cite[Page10]{barro1990}
%			\item parencite \\ \parencite{Greiner2008}
%			\item footcite \footcite{green2020}
%			\item \cite{Greiner2008}
%		\end{enumerate}
%		\begin{itemize}
%			\item \[V = \frac{4}{3}\pi r^3\]
%			\item $ V = \frac{4}{3}\pi r^3 $
%		\end{itemize}
%	\end{multicols}	
%	\begin{equation}
%		\label{eq1}
%		V = \frac{4}{3}\pi r^3
%	\end{equation}
%	\center 
%	As Equation(\ref{eq1}) shows, $\cdots$, this \emph{equation} is \alert{important}.
%\end{frame}
%
%\begin{frame}[noframenumbering, plain]{Snippets}
%	\begin{columns}
%		\column{0.5\textwidth}
%			\begin{block}{Remark}
%				Sample text
%			\end{block}
%			\begin{alertblock}{Important theorem}
%				Sample text in red box
%			\end{alertblock}
%			\begin{examples}
%				Sample text in green box. 
%			\end{examples}
%		\column{0.5\textwidth}
%			\begin{table}
%			    \centering
%			    \caption{table1}
%			    \vspace{-0.5cm}
%			    \setlength{\tabcolsep}{5mm}
%				    {
%				    \begin{tabular}{lcc}
%				    \hline
%			        123 & 123 & ad f \\ \hline
%			        \textcolor{deepred}{123} & w & ad f \\ 
%			        \textcolor{sufered}{123} & \alert{ad} f & ad s f \\ \hline
%				    \end{tabular}
%				    }
%			    \label{fig1}
%			\end{table}
%	\end{columns}
%\end{frame}
%\backupend
%%%%%%%%%%%%%%%%%%%%%%%%%%
\end{CJK*}
\end{document}
