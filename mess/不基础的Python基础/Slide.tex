\documentclass[UTF8, 16pt]{beamer}
 
% Chinese
\usepackage{CJKutf8}

% Font
\usepackage{bookman}
\usefonttheme{serif}
%\usepackage[T1]{fontenc}
%\usepackage{tgbonum}

% Other packages
\usepackage{hyperref, appendixnumberbeamer}
\usepackage{latexsym, amsmath, xcolor, multicol, booktabs}
\usepackage{graphicx, listings, stackengine}

% SUFE.sty
\usepackage{SUFE} 
% Bibtex
\usepackage[citestyle=authoryear-comp, 
			backend=bibtex, 
			bibstyle=numeric, 
%			sorting=ynt
			]{biblatex}
\setbeamertemplate{bibliography item}[text]
\addbibresource{ref.bib}

% Other setting


%%%%%%%%%%%%%%%%%%%%%%%%%%

% Title page
%% Author
\author[Haotian Deng] % The short name
{
Haotian Deng
%邓皓天
%\inst{1}
%\and
%Yuting Liu 
%\inst{2}
} 
%% Title & Subtitle
\title[Non-basic Python Basics]{Non-basic Python Basics}
%\subtitle{Subtitle}
%% Institution
\institute[SUFE]
{
%\inst{1}
Shanghai University of Finance and Economics
%上海财经大学金融学院
%\and
%\inst{2}
%Shanghai University of Finance and Economics
}
%% Date
\date[VLC 2021]
{\today}
%%Logo
%\logo{\includegraphics[height=1cm]{sufe_logo}}

%%%%%%%%%%%%%%%%%%%%%%%%%%

% Document begins
\begin{document}
\begin{CJK*}{UTF8}{gbsn}

%Title page
\begin{frame}[noframenumbering]
%	\thispagestyle{empty}
	\titlepage
	% Logo
	\vspace{-0.5cm}
    \begin{figure}[htpb] 
        \begin{center}
            \includegraphics[width=0.19 \linewidth]{sufe_logo.png}
        \end{center}  
    \end{figure}
\end{frame}

% Contents page
%\begin{frame}{Contents}
%	\tableofcontents[sectionstyle=show,
% 	subsectionstyle=show/shaded/hide,
% 	subsubsectionstyle=show/shaded/hide]
%\end{frame}

% Body

\begin{frame}{Why learning Python?}
	\begin{itemize}
		\item Easy to learn and quick to use, widely applicable, with numerous packages.
		\item Mindset
			\begin{itemize}
				\item The learning path of Python can be extrapolated to other fields.
				\item How to persist? interest and keeping using
				\item The belief that any specific problem can be solved given enough time.
			\end{itemize}
	\end{itemize}
\end{frame}
\begin{frame}{Four levels of Python Learning}
	\begin{enumerate}
		\item Basic Grammar
		\item Applications 
			\begin{itemize}
				\item Web Crawling
				\item Data Analysis
				\item Machine Learning
				\item ···
			\end{itemize}
		\item Framework
			\begin{itemize}
				\item Large Project
				\item Object-oriented Programming
			\end{itemize}
		\item The Zen of Python
	\end{enumerate}
\end{frame}

\begin{frame}{Preparation}
	How to ``write'' codes?
	\begin{itemize}
		\item Editor: VSCode or PyCharm
		\item ``.py'' V.S. ``.ipynb''
		\item Plugins
		\item Environment: Anaconda or Miniconda
		\item conda, pip, Pypi and GitHub
		\item LLM Assistance
		\item Comments and Annotations
		\item Remote SSH
		\item Write with others: git and code formatter
	\end{itemize}
\end{frame}

\begin{frame}{}
	\begin{itemize}
		\item How can one quickly and accurately find the desired information?
			\begin{enumerate}
				\item Browser and Extensions
				\item Google or Bing
				\item Bilibili and Xiaohongshu
				\item Zhihu and CSDN
			\end{enumerate}
		\item How to break down what you want into goals?
		\item How to lighten your own burden?
		\item Use Markdown or LaTex instead of Word.
	\end{itemize}
\end{frame}



%%%%%%%%%%%%%%%%%%%%%%%%%%

% End
\begin{frame}[allowframebreaks]%{End}
	\begin{center}
		\Huge\textbf{\textit{\texttt{Thanks!}}}
	\end{center}
\end{frame}

% Reference
%\appendix
%\begin{frame}{Reference}
%	\addtocounter{framenumber}{-1}
%	\printbibliography % [heading=bibintoc, title=Reference]
%\end{frame}

%%%%%%%%%%%%%%%%%%%%%%%%%%

% Snippets
%\begin{frame}[noframenumbering, plain]{Snippets}
%	\begin{multicols}{2}
%		\begin{enumerate}
%			\item \cite[Page10]{barro1990}
%			\item parencite \\ \parencite{Greiner2008}
%			\item footcite \footcite{green2020}
%			\item \cite{Greiner2008}
%		\end{enumerate}
%		\begin{itemize}
%			\item \[V = \frac{4}{3}\pi r^3\]
%			\item $ V = \frac{4}{3}\pi r^3 $
%		\end{itemize}
%	\end{multicols}	
%	\begin{equation}
%		\label{eq1}
%		V = \frac{4}{3}\pi r^3
%	\end{equation}
%	\center 
%	As Equation(\ref{eq1}) shows, $\cdots$, this \emph{equation} is \alert{important}.
%\end{frame}
%
%\begin{frame}[noframenumbering, plain]{Snippets}
%	\begin{columns}
%		\column{0.5\textwidth}
%			\begin{block}{Remark}
%				Sample text
%			\end{block}
%			\begin{alertblock}{Important theorem}
%				Sample text in red box
%			\end{alertblock}
%			\begin{examples}
%				Sample text in green box. 
%			\end{examples}
%		\column{0.5\textwidth}
%			\begin{table}
%			    \centering
%			    \caption{table1}
%			    \vspace{-0.5cm}
%			    \setlength{\tabcolsep}{5mm}
%				    {
%				    \begin{tabular}{lcc}
%				    \hline
%			        123 & 123 & ad f \\ \hline
%			        \textcolor{deepred}{123} & w & ad f \\ 
%			        \textcolor{sufered}{123} & \alert{ad} f & ad s f \\ \hline
%				    \end{tabular}
%				    }
%			    \label{fig1}
%			\end{table}
%	\end{columns}
%\end{frame}
%\backupend
%%%%%%%%%%%%%%%%%%%%%%%%%%
\end{CJK*}
\end{document}
