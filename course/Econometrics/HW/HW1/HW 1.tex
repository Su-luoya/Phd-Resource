\documentclass[12pt, a4paper, oneside]{article}
\usepackage{amsmath, amsthm, amssymb, bm, graphicx, hyperref, mathrsfs}

% Chinese
\usepackage{CJKutf8}

\title{\vspace{-3.3cm}\textbf{高级计量经济学\\第一次作业}}
\author{邓皓天\quad 2023310114}
\date{}
\linespread{1.5}
\newcounter{problemname}
\newcounter{answername}
\newenvironment{problem}{\stepcounter{problemname}\par\noindent\textbf{\arabic{problemname}、}}{\\\par}
\newenvironment{answer}{\stepcounter{answername}\par\noindent\textbf{答:}\newline}{\\\par}


\begin{document}
\begin{CJK*}{UTF8}{gbsn}
\maketitle
\begin{problem}
	在满足古典假设下,设定多元回归模型
	\vspace{-0.5cm}
	$$
	\vspace{-0.5cm}
	y_{i}=\beta_{0}+\beta_{1} x_{i, 1}+\beta_{2} x_{i, 2}+\cdots+\beta_{k} x_{i, k}+\mu_{i} \quad(i=1,2, \cdots, n)
	$$
	采用OLS对对以上模型进行参数估计,所得参数估计量记为$\boldsymbol{\hat{\beta}_n}$。
	现在在原来$n$个样本的情况下再增加一个样本,使得样本容量由原来的$n$变为现在的$n+1$,
	并再采用OLS对原始模型进行参数估计,所得参数估计量记为$\boldsymbol{\hat{\beta}_{n+1}}$。
	试问:
	\\
	(1)$\boldsymbol{\hat{\beta}_n}$和$\boldsymbol{\hat{\beta}_{n+1}}$相等吗?为什么?
	\\
	(2)如果$\boldsymbol{\hat{\beta}_n}$和$\boldsymbol{\hat{\beta}_{n+1}}$不相等,那么二者在满足什么样的条件下相等?
	\\
	(3)请讨论第$n$个样本点对$\boldsymbol{\hat{\beta}_n}$的影响。
\end{problem}
\begin{answer}
(1)$\boldsymbol{\hat{\beta}_n}$和$\boldsymbol{\hat{\beta}_{n+1}}$不一定相等。
当样本容量为$n$时,待估参数估计值的正规方程组为
$$
	\left\{\begin{array}{l}
	 n\hat{\beta}_{0}
	+\hat{\beta}_{1} \sum_{i=1}^n{X_{i, 1}}
	+\hat{\beta}_{2} \sum_{i=1}^n{X_{i, 2}}
	+\cdots
	+\hat{\beta}_{k} \sum_{i=1}^n{X_{i, k}}
	=\sum_{i=1}^n{Y_{i}}
	\\
	 \hat{\beta}_{0} \sum_{i=1}^n{X_{i, j}}
	+\hat{\beta}_{1} \sum_{i=1}^n{X_{i, j} X_{i, 1}}
	+\hat{\beta}_{2} \sum_{i=1}^n{X_{i, j} X_{i, 2}}
	\\+\cdots
	+\hat{\beta}_{k} \sum_{i=1}^n{X_{i, j} X_{i, k}}
	=\sum_{i=1}^n{X_{i, j}Y_{i}}
	\quad (j=1,2,\cdots,k)
	\end{array}\right.
$$
这$k+1$个方程组成的线性方程组可表示为以下矩阵形式:
$$
\left(\begin{array}{cccc}
	n & \sum X_{i, 1} & \cdots & \sum X_{i, k} 
	\\
	\sum X_{i, 1} & \sum X_{i, 1}^{2} & \cdots & \sum X_{i, 1} X_{i, k}
	\\
	\vdots & \vdots & & \vdots
	\\
	\sum X_{i k} & \sum X_{i k} X_{i 1} & \cdots & \sum X_{i, k}^{2}
\end{array}\right)
\left(\begin{array}{c}
	\hat{\beta}_{0}
	\\
	\hat{\beta}_{1}
	\\
	\vdots
	\\
	\hat{\beta}_{k}
\end{array}\right)
=
\left(\begin{array}{cccc}
	1 & 1 & \cdots & 1
	\\
	X_{1, 1} & X_{2, 1} & \cdots & X_{n, 1}
	\\
	\vdots & \vdots & & \vdots
	\\
	X_{1, k} & X_{2, k} & \cdots & X_{n k}
\end{array}\right)
\left(\begin{array}{c}
	Y_{1}\\ Y_{2} \\ \vdots \\ Y_{n}
\end{array}\right)
$$
新增一个样本点后,正规方程组变为
$$
	\left\{\begin{array}{l}
	 (n+\underline{1})\hat{\beta}_{0}
	+\hat{\beta}_{1} \left[ (\sum_{i=1}^n{X_{i, 1}}) + \underline{X_{(n+1), 1}} \right]
	+\hat{\beta}_{2} \left[ (\sum_{i=1}^n{X_{i, 2}}) + \underline{X_{(n+1), 2}} \right]
	\\+\cdots
	+\hat{\beta}_{k} \left[ (\sum_{i=1}^n{X_{i, k}}) + \underline{X_{(n+1), k}} \right]
	=\sum_{i=1}^n{Y_{i}}
	+\underline{Y_{n+1}}
	\\
	 \hat{\beta}_{0} \left[ (\sum_{i=1}^n{X_{i, j}}) + \underline{X_{(n+1), j}} \right]
	+\hat{\beta}_{1} \left[ (\sum_{i=1}^n{X_{i, j} X_{i, 1}}) + \underline{X_{(n+1), j} X_{(n+1), 1}} \right]
	\\+\hat{\beta}_{2} \left[ (\sum_{i=1}^n{X_{i, j} X_{i, 2}}) + \underline{X_{(n+1), j} X_{(n+1), 2}} \right]
	+\cdots
	+\hat{\beta}_{k} \left[ (\sum_{i=1}^n{X_{i, j} X_{i, k}}) + \underline{X_{(n+1), j} X_{(n+1), k}}  \right]
	\\=\sum_{i=1}^n{X_{ij}Y_{i}} + \underline{X_{(n+1),j} Y_{n+1}}
	\end{array}\right.
$$
同理,这$k+1$个方程组成的线性方程组可表示为矩阵形式:
	\vspace{-0.5cm}
	$$
	\vspace{-0.5cm}
	\left(\boldsymbol{X}^{\prime} \boldsymbol{X}\right) \boldsymbol{\hat{\beta}_{n+1}}=\boldsymbol{X}^{\prime} \boldsymbol{Y}
	$$
其中,$\left(\boldsymbol{X}^{\prime} \boldsymbol{X}\right)$为
$$
\left(\begin{array}{cccc}
	n+\underline{1} & (\sum_{i=1}^n{X_{i, 1}}) + \underline{X_{(n+1), 1}} & \cdots & (\sum_{i=1}^n{X_{i, k}}) + \underline{X_{(n+1), k}} 
	\\
	(\sum_{i=1}^n{X_{i, 1}}) + \underline{X_{(n+1), 1}} & \sum X_{i, 1}^{2} + \underline{X_{(n+1), 1}^2} & \cdots & \sum X_{i, 1} X_{i, k} + \underline{X_{(n+1), 1} X_{(n+1), k}}
	\\
	\vdots & \vdots & & \vdots
	\\
	(\sum_{i=1}^n{X_{i, k}}) + \underline{X_{(n+1), k}} & \sum X_{i, k} X_{i, 1} + \underline{X_{(n+1), k} X_{(n+1), 1}} & \cdots & \sum X_{i, k}^{2} + \underline{X_{(n+1), k}^{2}}
\end{array}\right)
$$
$$
\boldsymbol{\hat{\beta}_{n+1}} = 
\left(\begin{array}{c}
	\hat{\beta}_{0}
	\\
	\hat{\beta}_{1}
	\\
	\vdots
	\\
	\hat{\beta}_{k}
\end{array}\right)
,
\boldsymbol{X}^{\prime} \boldsymbol{Y}=
\left(\begin{array}{ccccc}
	1 & 1 & \cdots & 1 & \underline{1}
	\\
	X_{1, 1} & X_{2, 1} & \cdots & X_{n, 1} & \underline{X_{(n+1), 1}}
	\\
	\vdots & \vdots & & \vdots & \vdots
	\\
	X_{1, k} & X_{2, k} & \cdots & X_{n, k} & \underline{X_{(n+1), k}}
\end{array}\right)
\left(\begin{array}{c}
	Y_{1}\\ Y_{2} \\ \vdots \\ Y_{n} \\ \underline{Y_{(n+1)}}
\end{array}\right)
$$
显然,增加一个样本点后方程组的
系数矩阵$\left(\boldsymbol{X}^{\prime} \boldsymbol{X}\right)$
和增广矩阵$\boldsymbol{X}^{\prime} \boldsymbol{Y}$与之前均不同,
因此方程的解$\boldsymbol{\hat{\beta}_n}$和$\boldsymbol{\hat{\beta}_{n+1}}$也不一定相等。

\noindent
(2)若$\boldsymbol{\hat{\beta}_n} = \boldsymbol{\hat{\beta}_{n+1}}$,则两组正规方程组中的系数$\beta_{k}$均相等。将两组正规方程组相相减得
$$
\left\{\begin{array}{l}
	 \hat{\beta}_{0}
	+\hat{\beta}_{1} X_{(n+1), 1}
	+\hat{\beta}_{2} X_{(n+1), 2}
	+\cdots
	+\hat{\beta}_{k} X_{(n+1), k}
	=Y_{n+1}
	\\
	 \hat{\beta}_{0} X_{(n+1), j}
	+\hat{\beta}_{1} X_{(n+1), j} X_{(n+1), 1}
	+\hat{\beta}_{2} X_{(n+1), j} X_{(n+1), 2}
	\\+\cdots
	+\hat{\beta}_{k} X_{(n+1), j} X_{(n+1), k}
	=X_{(n+1),j} Y_{n+1}
	\quad (j=1,2,\cdots, k)
\end{array}\right.
$$
若$X_{(n+1), j}=0$,虽然增加前后的$\boldsymbol{X}^{\prime} \boldsymbol{Y}$结果未发生变化,但系数矩阵$\left(\boldsymbol{X}^{\prime} \boldsymbol{X}\right)$第一行第一列的元素从$n$变为$n+1$,因此这种情况下$\boldsymbol{\hat{\beta}_n} \neq \boldsymbol{\hat{\beta}_{n+1}}$,因此$X_{(n+1), j}\neq 0$。将上述方程组的第$2$至$k+1$个等式左右两边同除$X_{(n+1), j}$得
$$
\hat{\beta}_{0}
+\hat{\beta}_{1} X_{(n+1), 1}
+\hat{\beta}_{2} X_{(n+1), 2}
+\cdots
+\hat{\beta}_{k} X_{(n+1), k}
=Y_{n+1}
$$
显然,这也与上述方程组的第$1$个等式相同。
因此,若$\boldsymbol{\hat{\beta}_n} = \boldsymbol{\hat{\beta}_{n+1}}$,
则必然意味着新样本点$(X_{(n+1), j},Y_{(n+1)})$落在参数为$\boldsymbol{\hat{\beta}_n}$的样本回归线上。

\noindent
(3)若第$n$个样本点在样本回归线上,则其对$\boldsymbol{\hat{\beta}_n}$无影响;
	若第$n$个样本点在在样本回归线上侧,则样本回归线向上移动,靠近该样本点;
	反之同理。
\end{answer}

\begin{problem}
	在经典线性回归条件下,设
	$Y_{i}=\beta_{0}+\beta_{1} X_{i, 1}+\beta_{2} X_{i, 2}+\cdots+\beta_{k} X_{i, k}+\mu_{i} \quad(i=1,2, \cdots, n)$,
	采用 OLS 估计参数。模型的拟合值为$\hat{Y}_i$,
	设定拟合值与样本观测值的相关系数可以表示为:
	$$
	\rho = \frac{
	\sum{(Y_i-\bar{Y})(\hat{Y}_i-\bar{Y})}
	}{
	\sqrt{\sum{(Y_i-\bar{Y})^2}\sum{(\hat{Y}_i-\bar{Y})^2}}
	}
	$$
请推导$\rho$和拟合优度$R^2=1-\frac{RSS}{TSS}$的函数关系。
\end{problem}
\begin{answer}
欲证明
	$$
	\rho^2 =\frac{(\sum{y_i \hat{y}_i})^2}{\sum{y_i^2}\sum{\hat{y}_i^2}}
	 = 
	R^2 = 1-\frac{RSS}{TSS} = \frac{ESS}{TSS} = \frac{\sum{\hat{y}_i^2}}{\sum{y_i^2}}
	$$
即需证明
$$
\frac{(\sum{y_i \hat{y}_i})^2}{\sum{y_i^2}\sum{\hat{y}_i^2}} = \frac{\sum{\hat{y}_i^2}}{\sum{y_i^2}}
\Rightarrow
\sum{y_i \hat{y}_i} = \sum{\hat{y}_i^2}
$$
其中
$$
\sum{y_i \hat{y}_i} = \sum{(\hat{y}_i+e_i) \hat{y}_i} = \sum{\hat{y}_i^2} +\sum{e_i \hat{y}_i}
$$
而
$$
\sum{e_i \hat{y}_i} = \sum{e_i(\hat{Y}_i)-\bar{Y}}
 = \sum{e_i(\hat{\beta}_0 + \hat{\beta}_1X_{i,1} + \cdots +\hat{\beta}_k X_{i, k} - \bar{Y})}
$$
根据正规方程组可知$\sum{e_i}=0$,$\sum{e_i X_{i, j}}=0$,因此$\sum{e_i \hat{y}_i} = 0$。
所以
$$
\sum{y_i \hat{y}_i} = \sum{\hat{y}_i^2}
$$
得证
$$\rho^2 = R^2$$
\end{answer}

\begin{problem}
将拟合优度定义为:$R^2 = 1-\frac{RSS}{TSS}$。试问:

(1)$R^2$一定是在$ 0 $到$ 1 $之间吗?如果是,为什么?

(2)如果$R^2$不一定在$ 0 $到$ 1 $之间,请写出相关计量经济学模型表达式,并从理论上回答为什么?
\end{problem}
\begin{answer}
(1)$R^2$不一定在$ 0 $到$ 1 $之间。

\noindent	
(2)无截距项的回归模型的$R^2$不一定在$ 0 $到$ 1 $之间。
$$
\begin{aligned}
	R^2 & 
	= 1-\frac{RSS}{TSS} 
	= 1-\frac{\sum{(Y_i-\hat{Y}_i)^2}}{\sum{(Y_i-\bar{Y})^2}}
	= \frac{\sum{(Y_i-\bar{Y})^2}-\sum{(Y_i-\hat{Y}_i)^2}}{\sum{(Y_i-\bar{Y})^2}}
	\\
	&=\frac{(\sum{Y_i-\hat{Y}_i+\hat{Y}_i-\bar{Y})^2-\sum{(Y_i-\hat{Y}_i)^2}}}{\sum{(Y_i-\bar{Y})^2}}
	\\
	&=\frac{\sum\left(Y_{i}-\hat{Y}_{i}\right)^{2}+2 \sum\left(Y_{i}-\hat{Y}_{i}\right)\left(\hat{Y}_{i}-\bar{Y}\right)+\sum\left(\hat{Y}_{i}-\bar{Y}\right)^{2}-\sum\left(Y_{i}-\hat{Y}_{i}\right)^{2}}{\sum\left(Y_{i}-\bar{Y}\right)^{2}}
	\\
	&=\frac{2 \sum\left(Y_{i}-\hat{Y}_{i}\right)\left(\hat{Y}_{i}-\bar{Y}\right)+\sum\left(\hat{Y}_{i}-\bar{Y}\right)^{2}}{\sum\left(Y_{i}-\bar{Y}\right)^{2}}
	=\frac{2 \sum e_{i}\left(\hat{Y}_{i}-\bar{Y}\right)+\sum\left(\hat{Y}_{i}-\bar{Y}\right)^{2}}{\sum\left(Y_{i}-\bar{Y}\right)^{2}}
	\\
	&=\frac{2 \sum e_{i}\left(\hat{\beta}_{1} x_{i, 1}+\hat{\beta}_{2} x_{i, 2}+\cdots+\hat{\beta}_{k} x_{i, k}-\bar{Y}\right)+\sum\left(\hat{Y}_{i}-\bar{Y}\right)^{2}}{\sum\left(Y_{i}-\bar{Y}\right)^{2}}
	\\
	&=\frac{2\sum_{j=1}^k{\hat{\beta}_j \sum_{i=1}^n{e_i X_{i, j}}}-2\bar{Y}\sum{e_i}+\sum{(\hat{Y}_i-\bar{Y})^2}}{\sum{(Y_i-\bar{Y}})^2}
\end{aligned}
$$	
其中,仅有$\sum e_iX_{i, j}=0$,因此
$$
R^2 = \frac{-2\bar{Y}\sum{e_i}+\sum{(\hat{Y}_i-\bar{Y})^2}}{\sum{(Y_i-\bar{Y}})^2}
$$
当$2\bar{Y}\sum{e_i}>\sum{(\hat{Y}_i-\bar{Y})^2}$时,$R^2<0$;
当$\sum e_i(Y_i+\hat{Y}_i)<0$时,$R^2>1$
\end{answer}

\begin{problem}
在满足古典假设情况下,设定计量经济学模型:
$$Y_{i}=\beta_{0}+\beta_{1} X_{i, 1}+\beta_{2} X_{i, 2}+\cdots+\beta_{k} X_{i, k}+\mu_{i} \quad(i=1,2, \cdots, n)$$
如果参数$|\beta_2|>|\beta_1|$,那么$X_2$对于$Y$的影响大于$X_1$对于$Y$的影响,这种说法正确吗?
为什么?请从理论上加以回答。
\end{problem}
\begin{answer}
不正确。$|\beta_j| = \left|\frac{\partial y}{\partial x_{i, j}}\right|$代表在控制了其他解释变量后,$X_{i, j}$对$Y_i$的直接影响,而并未考虑其他间接影响。
也即某个变量可能通过其他变量间接对被解释变量产生影响,而这是无法单独通过参数$\beta$来表示的。另外,两个变量可能存在量纲不同的问题,不同的度量单位会导致$\beta$成比例变化。
\end{answer}
\end{CJK*}
\end{document}