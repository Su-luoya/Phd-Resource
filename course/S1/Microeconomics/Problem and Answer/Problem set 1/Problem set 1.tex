\documentclass[12pt,english]{article}
\usepackage{enumerate}
\usepackage[charter]{mathdesign}
\renewcommand{\familydefault}{\rmdefault}
\usepackage[T1]{fontenc}
\usepackage[latin9]{inputenc}
\usepackage[letterpaper]{geometry}
\usepackage{color}
\usepackage{babel}
\usepackage{mathrsfs}
\usepackage{amsthm}
\usepackage{amsmath}
\usepackage{setspace}
\usepackage[authoryear]{natbib}
\usepackage[all]{xy}
\onehalfspacing
\usepackage[unicode=true,pdfusetitle,
 bookmarks=true,bookmarksnumbered=false,bookmarksopen=false,
 breaklinks=false,pdfborder={0 0 1},backref=false,colorlinks=true]
 {hyperref}
\hypersetup{
 citecolor={darkblue}, urlcolor={darkblue}, linkcolor={darkblue}}

\makeatletter

%%%%%%%%%%%%%%%%%%%%%%%%%%%%%% LyX specific LaTeX commands.
\newcommand{\noun}[1]{\textsc{#1}}

%%%%%%%%%%%%%%%%%%%%%%%%%%%%%% User specified LaTeX commands.
\date{}
\bibpunct{(}{)}{,}{a}{,}{,} 
\usepackage{hyperref}
\let\ref\autoref
\usepackage{xcolor}
\definecolor{darkblue}{rgb}{0,0,0.5}
\usepackage[all]{xy}
\makeatletter
\newcommand{\xyC}[1]{%
\makeatletter
\xydef@\xymatrixcolsep@{#1}
\makeatother
} % end of \xyC
\makeatletter
\newcommand{\xyR}[1]{%
\makeatletter
\xydef@\xymatrixrowsep@{#1}
\makeatother
} % end of \xyR


\makeatother

\begin{document}

	
\begin{center}
\begin{large}
\textbf{Problem Set 1}
\end{large}
\bigskip

Due on September 21
\end{center}

\noindent\textbf{Problem 1.} 
Consider a new model of preferences, the \emph{$PI$-model}. The primitives of this model are two binary relations, $P$ and $I$, defined on $X$, where $P$ is interpreted as the "strictly better than" relation, and $I$ is interpreted as the "indifference" relation. We impose three conditions on $P$ and $I$ in this model: 
(1) for any $x\in X$, $xIx$  and $x\bar{P}x$; (2) for any $x,y\in X$ with $x\neq y$, \emph{exactly} one of the following three is true: $xPy,yPx$ and $xIy$; (3) both $P$ and $I$ are transitive. Based on the construction in this model, prove the following results. 

\noindent(a) $I$ is symmetric.  

\noindent(b) If $xPy$ and $yIz$, then $xPz$. If $xIy$ and $yPz$, then $xPz$.

\noindent (c) The $PI$-model is equivalent to the $\succeq$-model.  

\bigskip

\noindent\textbf{Problem 2.} Let $C$ be a choice correspondence
defined on the domain $\mathcal{D}$. Assume that for any $A,B\in \mathcal{D}$ with $A\cap B\neq \emptyset$, $A\cap B\in \mathcal{D}$. Show that if $C$
satisfies Sen's properties $\alpha$ and $\beta$, then 
$C$ satisfies the weak axiom of revealed preference. 


\bigskip

\noindent\textbf{Problem 3.} 
Let $\succeq$ be a preference relation defined on a \emph{finite} set $X$, and $\succ$ is the asymmetric 
component of $\succeq$. Notice that $\succeq$ is not assumed to be rational. We say $\succ$ is \emph{acyclic} if there does not exist a list $(x_{1},x_{2},...,x_{n-1},x_{n})$ such that $x_{k}\in X$ for each $k\in \left\{1,2,...,n\right\}$, $n\geq 2$, and 
$x_{1}\succ x_{2}\succ...\succ x_{n-1}\succ x_{n}\succ x_{1}$. 
For any $A\subseteq X$, let 
\[C_{\succ}(A)=\big\{x\in A: \text{there does not exist}\;\; y\in A \;\;\text{such that}\;\; y\succ x\big\}.\]
Prove the following results. 

\noindent (a)$C_{\succ}(A)\neq \phi$ for all non-empty  $A\subseteq X$ if and only if $\succ$ is acyclic. 

\noindent (b) Assume $\succ$ is acyclic. $C_{\succ}$ satisfies Sen's property $\alpha$, but may not satisfy property $\beta$. 

\bigskip

\noindent\textbf{Problem 4.} 
Show that if a choice correspondence $C$
(defined on the domain $\mathcal{D}$) can be rationalized, then it satisfies the \emph{path-invariance} property: for any $B_{1},B_{2}\in \mathcal{D}$ such that $B_{1}\cup B_{2}\in \mathcal{D}$ and $C(B_{1})\cup C(B_{2})\in \mathcal{D}$,
we have $$C(B_{1}\cup B_{2})= C\big(C(B_{1})\cup C(B_{2})\big).$$












  

\end{document}