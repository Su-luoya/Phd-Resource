\documentclass[12pt, a4paper, oneside]{article}
\usepackage{amsmath, amsthm, amssymb, bm, graphicx, hyperref, mathrsfs}

% Chinese
\usepackage{CJKutf8}

\title{\vspace{-3.3cm}\textbf{高级计量经济学\\第三次作业}}
\author{邓皓天\quad 2023310114}
\date{}
\linespread{1.5}
\newcounter{problemname}
\newcounter{answername}
\newenvironment{problem}{\stepcounter{problemname}\par\noindent\textbf{\arabic{problemname}、}}{\\}
\newenvironment{answer}{\stepcounter{answername}\par\noindent\textbf{答:}}{}


\begin{document}
\begin{CJK*}{UTF8}{gbsn}
\maketitle
\begin{problem}
	对于回归模型$Y_i=\beta_0+\beta_1 X_i+\mu_i$,如果随机解释变量$X$与随机误差项$\mu$相关,现存在$X$的一个工具变量$Z$。请求出采用工具变量法时(以$Z$作为工具变量)有关待估参数$\beta_1$的方差。
\end{problem}
\begin{answer}
	以$Z$作为工具变量得到的系数$\beta_1$的估计结果为
	$$
	\beta_1
	= \frac{\sum z_i y_i}{\sum z_i x_i}
	= \frac{\sum{z_i(Y_i-\bar{Y})}}{\sum z_i x_i}
	= \frac{\sum z_i Y_i}{\sum z_i x_i}-\frac{\bar{Y}\sum z_i}{\sum z_i x_i}
	= \frac{\sum z_i Y_i}{\sum z_i x_i}
	$$
	
	因此其方差为
	$$
	\begin{aligned}
		\text{Var}(\beta_1)
		&= \text{Var}(\frac{\sum z_i Y_i}{\sum z_i x_i} \mid X)
		= \sum(\frac{z_i}{\sum z_i x_i})^2\text{Var}\left[(\beta_0+\beta_1 X+\mu) \mid X\right]
		\\&= \sum(\frac{z_i}{\sum z_i x_i})^2 \cdot \sigma_{\mu}^2
		= \frac{\sum z_i^2}{(\sum z_i x_i)^2} \cdot \sigma_{\mu}^2
		\end{aligned}
	$$
	
	又因为$\sum x_i^2=n\sigma_{X}^2$,$\sum z_i^2=n\sigma_{Z}^2$,$(\sum z_i x_i)^2=n^2\rho_{X,Z}^2 \cdot \sigma_{X}^2 \cdot \sigma_{Z}^2$。所以
	$$\text{Var}(\beta_1)=\frac{\sigma_{\mu}^2}{n\sigma_{X}^2\rho_{X,Z}^2}$$
\end{answer}

\begin{problem}
	对于符合古典假设的回归模型$Y_i=\beta_0+\beta_1 X_i+\mu_i$而言,如果不存在$X$与随机误差项$\mu$相关,那么可以采用OLS对于参数$\beta_1$进行估计。如果同时在经济学理论上能够找到另外一个变量$Z$作为$X$的工具变量,那么也可以采用工具变量法对于参数$\beta_1$进行估计。试问,在以上情况下,即既可以采用OLS估计$\beta_1$,也可以采用IV(工具变量法)估计$\beta_1$的情况下,你认为该选择哪种方法来进行估计参数更加有优势?为什么?请加以证明。
\end{problem}
\begin{answer}
	如果模型满足古典假设,即误差项$\mu$与解释变量$X$是独立的,那么OLS估计量是最优的,因为在这种情况下,OLS估计量是BLUE(最佳线性无偏估计量)。这意味着在所有无偏的线性估计量中(包括IV估计的$\beta_1$),OLS估计量具有最小的方差。由于$\rho_{X,Z}^2 \in \left[0, 1\right]$,因此
	$$\text{Var}(\beta_1^{\text{OLS}})=\frac{\sigma_{\mu}^2}{n\sigma_{X}^2}
	\leqslant
	\frac{\sigma_{\mu}^2}{n\sigma_{X}^2\rho_{X,Z}^2}=\text{Var}(\beta_1^{\text{IV}})
	$$
\end{answer}

\begin{problem}
	对于回归模型$Y_i=\beta_0+\beta_1 X_{1, i}+\beta_2 X_{2, i}+\mu_i$,如果随机解释变量$X_2$与随机误差项$\mu$相关,请回答:
	
	(1)现找到$X_2$的一个工具变量$Z$。对于以上模型可以分别采用2SLS和IV法来进行参数估计,请分别写出参数估计量的表达式。在此基础上,试问2SLS和IV法所估计参数是等价的吗?为什么?
	
	(2)如果现在可以给$X_2$找到两个工具变量$Z_1$和$Z_2$,那么能够同时使用到两个工具变量的信息并采用IV法来对以上模型相关参数进行估计吗?为什么?如果不能,假设这时候可以选用任何一个工具变量的信息(采用IV法)来进行参数估计,请问IV所得结果与2SLS估计结果等价吗?为什么?请详细说明。
	
	(3)通过以上(1)和(2)的分析,你对于2SLS和IV法的相互关系有什么新发现?
\end{problem}
\begin{answer}
(1)此时2SLS和IV法所估计参数是等价的。

2SLS:首先对模型$Y_i=\beta_0+\beta_1 X_{1, i}+\beta_2 X_{2, i}+\mu_i$进行第一阶段的OLS估计,得到
$$\hat{X}_{2, i}=\gamma_{0}+\gamma_{1} X_{1, i}+\gamma_{2} Z_{i}+\varepsilon_{i}$$
用$\hat{X}_{2, i}$替换$X_{2, i}$得
$$Y_{i}=\beta_{0}+\beta_{1} X_{1, i}+\beta_{2} \hat{X}_{2, i}+\mu_{i}^{\prime}$$
对其进行第二阶段的OLS估计即可得到$\hat{\beta}^{\text{2SLS}}_2$。

IV法可直接得倒参数估计量
$$\hat{\beta}^{\text{IV}}_2=\frac{\text{Cov}(Y_i,Z_i)}{\text{Cov}(X_{2, i},Z_i)}$$

此时2SLS和IV法所估计的参数是等价的。因为在2SLS的第二阶段,我们实际上是在用$Z_i$作为$X_{2, i}$​的工具变量来进行估计,这与IV法的思想是一致的。

(2)可以将$Z_1$​和$Z_2$​合并为一个工具变量$Z$,然后采用$Z$来进行IV估计,即可同时使用到两个工具变量的信息。这种“合成”方法并不一定是2SLS中第一阶段的OLS,还可以是其他方法(例如机器学习等)。如果“合成”方法为OLS,则此时IV法与2SLS等价。

(3)2SLS可以看作是IV法的一个扩展,可以处理多个工具变量或者多个内生解释变量(过度识别)。而IV法则是更为基础的方法,它主要用于处理单个内生解释变量和单个工具变量(恰好识别)的情况。
\end{answer}
\end{CJK*}
\end{document}